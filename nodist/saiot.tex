\documentclass{scrartcl}
\usepackage{amsmath,amssymb,mathtools}
\usepackage[english]{babel}

% Allow to build the PDF with either lualatex (preferred) or pdflatex
\usepackage{iftex}
\ifLuaTeX
    \usepackage{luatextra}
\else
    \usepackage[T1]{fontenc}
    \usepackage[utf8]{inputenc}
    \usepackage{lmodern}
\fi

\title{S.A.I.O.T.}
\subtitle{Second Attempt to Improve the Offset Theory}
\author{Giandomenico Rossi \and Nicola Fontana}
\date{Oct 29, 2014}

\DeclarePairedDelimiter\abs{\lvert}{\rvert}%
\DeclarePairedDelimiter\norm{\lVert}{\rVert}%

\setlength\parindent{0pt}
\setlength\parskip{6pt}

\newcommand\V[1]{\vec{#1}}
\newcommand\D[1]{\dot{#1}}
\newcommand\DV[1]{\D{\V{#1}}}
\newcommand\SP[2]{\langle #1, #2 \rangle}

\begin{document}

\maketitle

\begin{abstract}
The results given by the BAIOCA algorithm in offsetting Bézier cubic
curves are clearly suboptimal. The fact that it minimizes the mean
square error between the original curve and the offset curve at the same
time value poses an artifical constraint (that \textbf{same time value}
thingy) that invalidate the whole theory. The optimal offset curve does
not necessarily need to have its own offset points at the same time
value of the original curve, and the fact that increasing the number of
samples decreases the quality of the approximation given by BAIOCA seems
to confirm that this is rarely the case.

This algorithm explores a new way to solve the system that uses
constraints not tied to the same time value.
\end{abstract}

\section{Mathematic}

The generic formula for a cubic Bézier curve is
\begin{equation*}
    \V{B}(t) =
	(1-t)^3 \V{B}_0 +
	3t (1-t)^2 \V{B}_1 +
	3t^2 (1-t) \V{B}_2 +
	t^3 \V{B}_3,
    \qquad t \in [ 0,1 ]
\end{equation*}
where $\V{B}_i \equiv ( B_{ix}, B_{iy} ) \in \mathbb{R} |_{i = 0,1,2,3}$
are known terms.

Let's call $\V{C}(t)$ the function that returns the offset of
$\V{B}(t)$ at the specific distance $R$, that is
\begin{equation}\label{eq:C}
\V{C}(t) = \V{B}(t) + R
    \frac{(\D{B}(t)_y, -\D{B}(t)_x)}{\sqrt{\D{B}(t)_x^2 + \D{B}(t)_y^2}},
    \qquad t \in [ 0,1 ]
\end{equation}
where $\DV{B}(t) \equiv ( \D{B}(t)_x, \D{B}(t)_y ) \equiv \left( \frac{d}{dt}
B_x(t), \frac{d}{dt} B_y(t) \right)$.

We must find a Bézier curve
\begin{equation}\label{eq:Q}
    \V{Q}(t) =
	(1-t)^3 \V{Q}_0 +
	3t (1-t)^2 \V{Q}_1 +
	3t^2 (1-t) \V{Q}_2 +
	t^3 \V{Q}_3
\end{equation}
where $\V{Q}_i \equiv ( Q_{ix}, Q_{iy} ) \in \mathbb{R} |_{i = 0,1,2,3}$
are the terms that must be found.

The offset curve must respect the constraints described in the abstract
of the BAIOCA algorithm, that is
\begin{enumerate}
\item $\V{Q}(0) = \V{C}(0)$ and $\V{Q}(1) = \V{C}(1)$ (interpolation);
\item $\DV{Q}(0) = \DV{B}(0)$ and $\DV{Q}(1) = \DV{B}(1)$ (tangents),
where $\DV{Q}(t) \equiv \frac{d}{dt} \V{Q}(t)$.
\end{enumerate}

\medskip
Condition 1 implies that $\V{Q}_0$ and $\V{Q}_3$ can be calculated by
directly using \eqref{eq:C}.

Condition 2 implies that $\DV{Q}_0 = \DV{B}_0$ and
$\DV{Q}_3 = \DV{B}_3$. Imposing by convention
\begin{equation}\label{eq:pi}
    \V{P}_i = \V{B}_{i+1} - \V{B}_i, \qquad i = 0, 1, 2
\end{equation}
we can calculate $\DV{B}_0$ and $\DV{B}_3$ directly from the hodograph
of $\V{B}(t)$:
\begin{align*}
    \DV{B}(t) &= 3 \left[ (1-t)^2 \V{P}_0 + 2t(1-t) \V{P}_1 + t^2 \V{P}_2 \right]; \\
    \DV{B}(0) &= 3 \V{P}_0 \equiv \DV{B}_0 = \DV{Q}_0 \\
    \DV{B}(1) &= 3 \V{P}_2 \equiv \DV{B}_3 = \DV{Q}_3.
\end{align*}
Knowing that one of the properties of a Bézier curve is the start of the
curve is tangent to the first section of the control polygon and the end
is tangent to the last section, condition 2 is hence equivalent to:
\begin{equation*}
\begin{aligned}[t]
    \V{Q}_1 &= \V{Q}_0 + \frac{r}{3} \DV{Q}_0 = \V{Q}_0 + r \V{P}_0, \\
    \V{Q}_2 &= \V{Q}_3 + \frac{s}{3} \DV{Q}_3 = \V{Q}_3 + s \V{P}_2.
\end{aligned}
\qquad r, s \in \mathbb{R}
\end{equation*}

Substituting in \eqref{eq:Q} we get
\begin{equation}\label{eq:QQ}
    \V{Q}(t) =
	(1-t)^3 \V{Q}_0 +
	3t (1-t)^2 ( \V{Q}_0 + r \V{P}_0 ) +
	3t^2 (1-t) ( \V{Q}_3 + s \V{P}_2 ) +
	t^3\V{Q}_3.
\end{equation}

The hodograph of $\V{Q}(t)$ is then
\begin{equation}\label{eq:QD}
    \DV{Q}(t) = 3 \left[
	(1-t)^2 r \V{P}_0 +
	2t(1-t) (\V{Q}_3 + s \V{P}_2 - \V{Q}_0 - r \V{P}_0) -
	t^2 s \V{P}_2
    \right].
\end{equation}

We must find a couple of values for $r$ and $s$ that gives us a decent
offset curve in most situations. Put in other words, we need to impose
some additional constraint to be able to solve the system.

Let consider a generic point in $\V{B}(t)$ at $t=m$ (where $m$ is chosen
in some manner, typically $0.5$). Let impose that the offset curve passes
throught $\V{C}(m)$ and let call $u$ the $t$ value\footnote{$m \neq u$
is the original assumption that supposedly gives better results than the
BAIOCA algorithm.} in which $\V{Q}(t) = \V{C}(m)$. Then, let impose the
derivative in $\V{Q}(u)$ is equal to the derivative in $\V{B}(m)$.

The above constraints can be expressed by the following system with 4
equations and 3 variables ($r$, $s$ and $u$).
\begin{equation*}
\begin{dcases}
    \V{Q}(u) = \V{B}(m)\\
    \DV{Q}(u) = \DV{B}(m).
\end{dcases}
\end{equation*}

The second condition, according to \eqref{eq:QD}, is equivalent to
\begin{equation*}
\begin{aligned}
    (1-u)^2 r \V{P}_0 +
	2u(1-u) (\V{Q}_3 + s \V{P}_2 - \V{Q}_0 - r \V{P}_0) -
	u^2 s \V{P}_2 =
	\frac{\DV{B}(m)}{3}; \\
    3 \V{P}_0 (1-u)(1-3u) r + 3 \V{P}_2 u(2-3u) s =
	\DV{B}(m) - 6u(1-u) (\V{Q}_3 - \V{Q}_0).
\end{aligned}
\end{equation*}

By developing the system in $x$ and $y$
\begin{equation*}
\begin{dcases}
    3 P_{0x} (1-u) (1-3u) r + 3 P_{2x} u (2-3u) s = \D{B}_x(m) - 6u (1-u) (Q_{3x} - Q_{0x}) \\
    3 P_{0y} (1-u) (1-3u) r + 3 P_{2y} u (2-3u) s = \D{B}_y(m) - 6u (1-u) (Q_{3y} - Q_{0y})
\end{dcases}
\end{equation*}
we can find $r$ and $s$ in $u$:
\begin{equation*}
\begin{dcases}
    r &= \frac{\D{B}_x(m) P_{2y} - \D{B}_y(m) P_{2x} + 6u (1-u) [ P_{2x} (Q_{3y} - Q_{0y}) - P_{2y} (Q_{3x} - Q_{0x}) ]}
	{3 (1-u) (1-3u) ( P_{0x} P_{2y} - P_{2x} P_{0y} )} \\
    s &= \frac{\D{B}_y(m) P_{0x} - \D{B}_x(m) P_{0y} + 6u (1-u) [ P_{0y} (Q_{3x} - Q_{0x}) - P_{0x} (Q_{3y} - Q_{0y}) ]}
	{3u (2-3u) ( P_{0x} P_{2y} - P_{2x} P_{0y} )}
\end{dcases}
\end{equation*}
from which we derive the following known terms:
\begin{equation}\label{eq:terms}
\begin{aligned}[t]
    A_1 &= P_{0x} P_{2y} - P_{2x} P_{0y}; \\
    A_2 &= \D{B}_x(m) P_{2y} - \D{B}_y(m) P_{2x}; \\
    A_3 &= 6 [ P_{2x} (Q_{3y} - Q_{0y}) - P_{2y} (Q_{3x} - Q_{0x}) ]; \\
    A_4 &= \D{B}_y(m) P_{0x} - \D{B}_x(m) P_{0y}; \\
    A_5 &= 6 [ P_{0y} (Q_{3x} - Q_{0x}) - P_{0x} (Q_{3y} - Q_{0y}) ].
\end{aligned}
\end{equation}

The above system can hence be rewritten as
\begin{equation}\label{eq:rs}
\begin{dcases}
    r &= \frac{A_2 + A_3 u (1-u)}{3 A_1 (1-u) (1-3u)} \\
    s &= \frac{A_4 + A_5 u (1-u)}{3 A_1 u (2-3u)}
\end{dcases}
\end{equation}

Now to calculate $u$ we can apply one equation from the first condition.

From \eqref{eq:QQ} we know that
\begin{equation*}
    \V{Q}(u) =
	(1-u)^3 \V{Q}_0 +
	3u (1-u)^2 \V{Q}_0 +
	3u^2 (1-u) \V{Q}_3 +
	u^3 \V{Q}_3 +
	3u (1-u)^2 \V{P}_0 r +
	3u^2 (1-u) \V{P}_2 s.
\end{equation*}

Substituting $r$ and $s$ from \eqref{eq:rs} brings us to
\begin{equation*}
\begin{split}
    \V{Q}(u) &= (1-u)^3 \V{Q}_0 + 3u (1-u)^2 \V{Q}_0 + u^3 \V{Q}_3 + 3u^2 (1-u) \V{Q}_3 \\
     & \quad + u (1-u) \V{P}_0 \frac{A_2 + A_3 u (1-u)}{A_1 (1-3u)}
	+ u (1-u) \V{P}_2 \frac{A_4 + A_5 u (u-1)}{A_1 (2-3u)}.
\end{split}
\end{equation*}

\end{document}
