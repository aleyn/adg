\documentclass{scrartcl}
\usepackage{amsmath,amssymb,mathtools}
\usepackage[english]{babel}

% Allow to build the PDF with either lualatex (preferred) or pdflatex
\usepackage{iftex}
\ifLuaTeX
    \usepackage{luatextra}
\else
    \usepackage[T1]{fontenc}
    \usepackage[utf8]{inputenc}
    \usepackage{lmodern}
\fi

\title{Handcraft algorithm}
\subtitle{A disjointed algorithm for offsetting cubic Bézier curves}
\author{Nicola Fontana}
\date{February 25, 2009}

\setlength\parindent{0pt}
\setlength\parskip{6pt}

\begin{document}

\maketitle

\begin{abstract}
This is basically the transcription of the algorithm used by the
ADG canvas on its early development phases. This algorithm has been
superseded in recent releases by BAIOCA. It is included here for
completeness and historical reasons.
\end{abstract}

Given a cubic Bézier primitive, it must be found the approximated
Bézier curve parallel to the original one at a specific distance
(the so called "offset curve"). The four points needed to build
the new curve must be returned.

To solve the offset problem, a custom algorithm is used. First, the
resulting curve MUST have the same slope at the start and end point.
These constraints are not sufficient to resolve the system, so I let
the curve pass thought a given point (r, known and got from the
original curve) at a given time (m, now hardcoded to 0.5).

Firstly, some useful variables are defined:

\begin{equation*}
\begin{split}
    v_0 &= \text{unitvector}(\text{p[1]} - \text{p[0]}) \times \text{offset} \\
    v_3 &= \text{unitvector}(\text{p[3]} - \text{p[2]}) \times \text{offset} \\
    p_0 &= \text{p[0]} + \text{normal}(v_0) \\
    p_3 &= \text{p[3]} + \text{normal}(v_3) \\
\end{split}
\end{equation*}

The resulting curve must have the same slopes than the original
one at the start and end points. Forcing the same slopes means:

\begin{equation*}
p_1 = p_0 + k_0 v_0.
\end{equation*}

where $k_0$ is an arbitrary factor. Decomposing for $x$ and $y$:

\begin{equation*}
\begin{dcases}
    p_{1x} = p_{0x} + k_0 v_{0x} \\
    p_{1y} = p_{0y} + k_0 v_{0y} \\
\end{dcases}
\end{equation*}

and doing the same for the end point:

\begin{equation*}
\begin{dcases}
    p_{2x} = p_{3x} + k_3 v_{3x} \\
    p_{2y} = p_{3y} + k_3 v_{3y} \\
\end{dcases}
\end{equation*}

This does not give a resolvable system though. The curve will be
interpolated by forcing its path to pass through $r$ when
\textit{time} is $m$, where $0 \leq m \leq 1$. Knowing the function of
the cubic Bézier:

\begin{equation*}
C(t) = (1-t)^3p_0 + 3t(1-t)^2p_1 + 3t^2(1-t)p_2 + t^3p_3.
\end{equation*}

and forcing $t = m$ and $C(t) = r$:

\begin{equation*}
\begin{split}
    r &= (1-m)^3 p_0 + 3m(1-m)^2 p_1 + 3m^2 (1-m) p_2 + m^3 p_3 \\
    (1-m) p_1 + m p_2 &= \frac{r - (1-m)^3p_0 - m^3p_3}{3m (1-m)} \\
\end{split}
\end{equation*}

brings to the final system:

\begin{equation*}
\begin{dcases}
    p_{1x} = p_{0x} + k_0 v_{0x} \\
    p_{1y} = p_{0y} + k_0 v_{0y} \\
    p_{2x} = p_{3x} + k_3 v_{3x} \\
    p_{2y} = p_{3y} + k_3 v_{3y} \\
    (1-m) p_{1x} + m p_{2x} = \frac{r_x - (1-m)^3 p_{0x} - m^3 p_{3x}}{3m (1-m)} \\
    (1-m) p_{1y} + m p_{2y} = \frac{r_y - (1-m)^3 p_{0y} - m^3 p_{3y}}{3m (1-m)} \\
\end{dcases}
\end{equation*}

Substituting and resolving for $k_0$ and $k_3$:

\begin{equation*}
\begin{dcases}
    (1-m) k_0 v_{0x} + m k_3 v_{3x} = \frac{r_x - (1-m)^3 p_{0x} - m^3 p_{3x}}{3m (1-m)} - (1-m) p_{0x} - m p_{3x} \\
    (1-m) k_0 v_{0y} + m k_3 v_{3y} = \frac{r_y - (1-m)^3 p_{0y} - m^3 p_{3y}}{3m (1-m)} - (1-m) p_{0y} - m p_{3y} \\
\end{dcases}
\end{equation*}

\begin{equation*}
\begin{dcases}
    (1-m) k_0 v_{0x} + m k_3 v_{3x} = \frac{r_x - (1-m)^2 (1+2m) p_{0x} - m^2 (3-2m) p_{3x}}{3m (1-m)} \\
    (1-m) k_0 v_{0y} + m k_3 v_{3y} = \frac{r_y - (1-m)^2 (1+2m) p_{0y} - m^2 (3-2m) p_{3y}}{3m (1-m)} \\
\end{dcases}
\end{equation*}

Letting:

\begin{equation*}
    s = \frac{r - (1-m)^2(1+2m) p_0 - m^2(3-2m) p_3}{3m (1 - m)}
\end{equation*}

reduces the above to this final equations:

\begin{equation*}
\begin{dcases}
    s_x = (1-m) k_0 v_{0x} + m k_3 v_{3x} \\
    s_y = (1-m) k_0 v_{0y} + m k_3 v_{3y} \\
\end{dcases}
\end{equation*}

If $v_{0x} \ne; 0$, the system can be resolved for $k_0$ and $k_3$
calculated accordingly:

\begin{equation*}
\begin{dcases}
    k_0 = \frac{s_x - m k_3 v_{3x}}{(1-m) v_{0x}} \\
    s_y = \frac{(s_x - m k_3 v_{3x}) v_{0y}}{v_{0x}} + m k_3 v_{3y} \\
\end{dcases}
\end{equation*}

\begin{equation*}
\begin{dcases}
    k_0 = \frac{s_x - m k_3 v_{3x}}{(1-m) v_{0x}} \\
    s_y - \frac{s_x v_{0y}}{v_{0x}} = k_3 m (v_{3y} - \frac{v_{3x} v_{0y}}{v_{0x}}) \\
\end{dcases}
\end{equation*}

\begin{equation*}
\begin{dcases}
    k_0 = \frac{s_x - m k_3 v_{3x}}{(1-m) v_{0x}} \\
    k_3 = \frac{s_y - s_x \frac{v_{0y}}{v_{0x}}}{m (v_{3y} - v_{3x} \frac{v_{0y}}{v_{0x}})} \\
\end{dcases}
\end{equation*}

Otherwise, if $v_{3x} \ne 0$, the system can be solved for $k_3$ and $k_0$
calculated accordingly:

\begin{equation*}
\begin{dcases}
    k_3 = \frac{s_x - (1-m) k_0 v_{0x}}{m v_{3x}} \\
    s_y = (1-m) k_0 v_{0y} + \frac{[s_x - (1-m) k_0 v_{0x}] v_{3y}}{v_{3x}} \\
\end{dcases}
\end{equation*}

\begin{equation*}
\begin{dcases}
    k_3 = \frac{s_x - (1-m) k_0 v_{0x}}{m v_{3x}} \\
    k_0 (1-m) (v_{0y} - k_0 v_{0x} \frac{v_{3y}}{v_{3x}}) = s_y - s_x \frac{v_{3y}}{v_{3x}} \\
\end{dcases}
\end{equation*}

\begin{equation*}
\begin{dcases}
    k_3 = \frac{s_x - (1-m) k_0 v_{0x}}{m v_{3x}} \\
    k_0 = \frac{s_y - s_x \frac{v_{3y}}{v_{3x}}}{(1-m) (v_{0y} - v_{0x} \frac{v_{3y}}{v_{3x}})} \\
\end{dcases}
\end{equation*}

The whole process must be guarded against division by 0 exceptions.
If either $v_{0x}$ and $v_{3x}$ are 0, the first equation will be
inconsistent. More in general, the
$v_{0x} \times v_{3y} = v_{3x} \times v_{3y}$ condition must be avoided. This is
the first situation to avoid, in which case an alternative approach
should be used.

\end{document}
