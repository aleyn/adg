\documentclass{scrartcl}
\usepackage{amsmath,amssymb,mathtools}
\usepackage[english]{babel}

% Allow to build the PDF with either lualatex (preferred) or pdflatex
\usepackage{iftex}
\ifLuaTeX
    \usepackage{luatextra}
\else
    \usepackage[T1]{fontenc}
    \usepackage[utf8]{inputenc}
    \usepackage{lmodern}
\fi

\title{B.A.I.O.C.A.}
\subtitle{Bare Attempt to Improve Offset Curve Algorithm}
\author{Giandomenico Rossi \and Nicola Fontana}
\date{April 16, 2013}

\DeclarePairedDelimiter\abs{\lvert}{\rvert}%
\DeclarePairedDelimiter\norm{\lVert}{\rVert}%

\setlength\parindent{0pt}
\setlength\parskip{6pt}

\newcommand\V[1]{\vec{#1}}
\newcommand\D[1]{\dot{#1}}
\newcommand\DV[1]{\D{\V{#1}}}
\newcommand\SP[2]{\langle #1, #2 \rangle}

\begin{document}

\maketitle

\begin{abstract}
Computer Aided Design (CAD) and related software is often based on cubic
Bézier curves: the Postscript language and consecuentely the PDF file
format are two widespread examples of such software. Defining an optimal
algorithm for approximating a Bézier curve parallel to the original one
at a specific distance (the so called ``offset curve'') is a big
requirement in CAD drafting: it is heavily used while constructing
derived entities (e.g., a fillet) or to express machining allowance.

This document describes an algorithm suitable for CAD purposes. In those
cases, the starting and ending points of the offset curve \textbf{must}
have coordinates and slopes coincident with the perfect solution, so the
continuity with previous and next offseted entity is preserved.
\end{abstract}

\section{Mathematic}

The generic formula for a cubic Bézier curve is
\begin{equation*}
\V{B}(t) = b_0(t)\V{B}_0 + b_1(t)\V{B}_1 + b_2(t)\V{B}_2 + b_3(t)\V{B}_3
\end{equation*}
where
\begin{equation*}
\begin{split}
\V{B}_i &\equiv ( B^i_x, B^i_y ) \in \mathbb{R}; \qquad i = 0,1,2,3 \\
b_i(t)  &\equiv \binom{3}{i} t^i (1-t)^{3-i}. \\
\end{split}
\end{equation*}

Given in $\{t_i\}_{i=0}^n$ a set of values for $t$ chosen in some
manner with $t_0=0, t_n=1$ and in $R$ the required distance of the offset
curve,
\begin{equation}\label{eq:Ci}
\V{C}_i = \V{B}(t_i) + R \left.
\frac{\D{B}_y, -\D{B}_x}{\sqrt{\D{B}_x^2 + \D{B}_y^2}}
    \,\right|_{t=t_i} \forall t_i
\end{equation}
is the equation of the offset curve that has in $\{\V{C}_i\}_{i=0}^n$ the
set of its points and where $\DV{B} \equiv ( \D{B}_x, \D{B}_y ) \equiv
\left( \frac{d}{dt} B_x(t), \frac{d}{dt} B_y(t) \right)$.

We must find the Bézier curve
\begin{equation}\label{eq:q}
\V{Q}(t) = b_0(t) \V{Q}_0 + b_1(t) \V{Q}_1 + b_2(t) \V{Q}_2 + b_3(t) \V{Q}_3
\end{equation}
where
\begin{equation*}
\V{Q}_i \equiv ( Q^i_x, Q^i_y ) \in \mathbb{R} \qquad i = 0,1,2,3
\end{equation*}
which best fits \eqref{eq:Ci} within the needed constraints, that is:
\begin{enumerate}
\item $\V{Q}(0) = \V{C}_0$ and $\V{Q}(1) = \V{C}_n$ (interpolation);
\item $\DV{Q}(0) = \DV{B}(0)$ and $\DV{Q}(1) = \DV{B}(1)$ (tangents),
where $\DV{Q} \equiv \frac{d}{dt} \V{Q}(t)$.
\end{enumerate}

\medskip
Condition 1 implies that
\begin{equation}\label{eq:q03}
\begin{split}
    \V{Q}_0 &= \V{B}_0 + R \frac{\D{B}_{0y}, -\D{B}_{0x}}
	{\| \DV{B}_0 \|}; \qquad
	\| \DV{B}_i \| = \sqrt{\D{B}_{ix}^2 + \D{B}_{iy}^2} \\
    \V{Q}_3 &= \V{B}_3 + R \frac{\D{B}_{3y}, -\D{B}_{3x}}
	{\| \DV{B}_3 \|}. \\
\end{split}
\end{equation}

Condition 2 implies that $\DV{Q}(0) \parallel \DV{B}(0)$ and that
$\DV{Q}(1) \parallel \DV{B}(1)$.
Given that
\begin{equation}
\begin{split}
    \DV{B}(t) &= 3 \left(
	g_0(t) \Delta \V{B}_0 +
	g_1(t) \Delta \V{B}_1 +
	g_2(t) \Delta \V{B}_2
    \right). \qquad \Delta \V{B}_i = \V{B}_{i+1} - \V{B}_i \\
    \DV{Q}(0) &= 3 ( \V{Q}_1 - \V{Q}_0 ) = 3 ( \V{B}_1 - \V{B}_0 ); \\
    \DV{Q}(1) &= 3 ( \V{Q}_2 - \V{Q}_3 ) = 3 ( \V{B}_2 - \V{B}_3 ).
\end{split}
\end{equation}

Condition 2 is hence equivalent to
\begin{equation}\label{eq:q12}
\begin{split}
    \V{Q}_1 &= \V{Q}_0 + r \Delta \V{B}_0 \qquad r \in \mathbb{R}; \\
    \V{Q}_2 &= \V{Q}_3 - s \Delta \V{B}_2 \qquad s \in \mathbb{R}.
\end{split}
\end{equation}

Substituting \eqref{eq:q12} in \eqref{eq:q} we get
\begin{equation*}
\V{Q}(t) = b_0(t) \V{Q}_0 + b_1(t) \V{Q}_0 + b_1(t) r \Delta \V{B}_0 +
b_2(t) \V{Q}_3 - b_2(t) s \Delta \V{B}_2 + b_3(t) \V{Q}_3.
\end{equation*}

Determine the value of $r$ and $s$ that minimizes the quantity
$\phi = \sum \left[ \V{C}_i - \V{Q}(t_i) \right]^2$, equivalent to
solve the system
\begin{equation*}
\begin{dcases}
\frac{\delta\phi}{\delta r} = 0;\\
\frac{\delta\phi}{\delta s} = 0.
\end{dcases}
\end{equation*}

Now, given the shortcuts
\begin{equation*}
\begin{split}
    \sum &\equiv \sum_{i=0}^{n}; \\
    b_j &\equiv b_j(t_i); \\
\end{split}
\end{equation*}
we can develop further to get
\begin{equation*}
    \phi(r, s) = \sum \left( \V{C}_i - b_0 \V{Q}_0 - b_1 \V{Q}_0 -
    b_1 r \Delta \V{B}_0 - b_2 \V{Q}_3 + b_2 s \Delta \V{B}_2 - b_3 \V{Q}_3 \right)^2;
\end{equation*}

that, applied to the previous system, brings us to the following linear system
\begin{equation*}
\begin{dcases}
    2 \sum \left(
	\V{C}_i - b_0 \V{Q}_0 - b_1 \V{Q}_0 - r b_1 \Delta \V{B}_0 -
	b_2 \V{Q}_3 + s b_2 \Delta \V{B}_2 - b_3 \V{Q}_3
    \right) \left( b_1 \Delta \V{B}_0 \right) &= 0; \\
    2 \sum \left(
	\V{C}_i - b_0 \V{Q}_0 - b_1 \V{Q}_0 - r b_1 \Delta \V{B}_0 -
	b_2 \V{Q}_3 + s b_2 \Delta \V{B}_2 - b_3 \V{Q}_3
    \right) \left( -b_2 \Delta \V{B}_2 \right) &= 0; \\
\end{dcases}
\end{equation*}
from which, considering $\V{P}_i \equiv \Delta \V{B}_i$, we get
\begin{equation*}
\begin{dcases}
    \sum \left(
    \begin{split}
	b_1       \SP{\V{C}_i}{\V{P}_0} -
	b_0 b_1   \SP{\V{Q}_0}{\V{P}_0} -
	b_1 b_1   \SP{\V{Q}_0}{\V{P}_0} -
	r b_1 b_1 \SP{\V{P}_0}{\V{P}_0} \\ -
	b_1 b_2   \SP{\V{Q}_3}{\V{P}_0} +
	s b_1 b_2 \SP{\V{P}_2}{\V{P}_0} -
	b_1 b_3   \SP{\V{Q}_3}{\V{P}_0}
    \end{split}
    \right) &= 0; \\
    \sum \left(
    \begin{split}
	b_2       \SP{\V{C}_i}{\V{P}_2} -
	b_0 b_2   \SP{\V{Q}_0}{\V{P}_2} -
	b_1 b_2   \SP{\V{Q}_0}{\V{P}_2} -
	r b_1 b_2 \SP{\V{P}_0}{\V{P}_2} \\ -
	b_2 b_2   \SP{\V{Q}_3}{\V{P}_2} +
	s b_2 b_2 \SP{\V{P}_2}{\V{P}_2} -
	b_2 b_3   \SP{\V{Q}_3}{\V{P}_2}
    \end{split}
    \right) &= 0. \\
\end{dcases}
\end{equation*}

Given the additional conventions
\begin{equation}\label{eq:DE}
\begin{split}
    D_0 &\equiv \sum b_1 \SP{\V{C}_i}{\V{P}_0}; \\
    D_2 &\equiv \sum b_2 \SP{\V{C}_i}{\V{P}_2}; \\
    E_{jk} &\equiv \sum b_j b_k; \\
\end{split}
\end{equation}
we can substitute to get
\begin{equation*}
\begin{dcases}
\begin{split}
    D_0 - E_{01}   \SP{\V{Q}_0}{\V{P}_0} -
	  E_{11}   \SP{\V{Q}_0}{\V{P}_0} -
	  r E_{11} \SP{\V{P}_0}{\V{P}_0} -
	  E_{12}   \SP{\V{Q}_3}{\V{P}_0} \\ +
	  s E_{12} \SP{\V{P}_2}{\V{P}_0} -
	  E_{13}   \SP{\V{Q}_3}{\V{P}_0} &= 0;
\end{split} \\
\begin{split}
    D_2 - E_{02}   \SP{\V{Q}_0}{\V{P}_2} -
	  E_{12}   \SP{\V{Q}_0}{\V{P}_2} -
	  r E_{12} \SP{\V{P}_0}{\V{P}_2} -
	  E_{22}   \SP{\V{Q}_3}{\V{P}_2} \\ +
	  s E_{22} \SP{\V{P}_2}{\V{P}_2} -
	  E_{23}   \SP{\V{Q}_3}{\V{P}_2} &= 0;
\end{split} \\
\end{dcases}
\end{equation*}
and derive the following known terms
\begin{equation}\label{eq:ACD}
\begin{split}
A_1 &= D_0 -
    \SP{\V{Q}_0}{\V{P}_0} (E_{01} + E_{11}) -
    \SP{\V{Q}_3}{\V{P}_0} (E_{12} + E_{13}); \\
A_2 &= D_2 -
    \SP{\V{Q}_0}{\V{P}_2} (E_{02} + E_{12}) -
    \SP{\V{Q}_3}{\V{P}_2} (E_{22} + E_{23}); \\
A_3 &= \SP{\V{P}_0}{\V{P}_0} E_{11}; \\
A_4 &= \SP{\V{P}_0}{\V{P}_2} E_{12}; \\
A_5 &= \SP{\V{P}_2}{\V{P}_2} E_{22}. \\
\end{split}
\end{equation}

The system is hence reduced to
\begin{equation*}
\begin{dcases}
    s A_4 - r A_3 &= A_1; \\
    s A_5 - r A_4 &= A_2; \\
\end{dcases}
\end{equation*}
from which we can calculate $r$ and $s$
\begin{equation}\label{eq:rs}
\begin{split}
    r &= \frac{A_2 A_4 - A_1 A_5}{A_3 A_5 - A_4 A_4}; \\
    s &= \frac{A_2 A_3 - A_1 A_4}{A_3 A_5 - A_4 A_4}. \\
\end{split}
\end{equation}

\section{Algorithm}

\begin{enumerate}
    \item Select $\{t_i\}_{i=0}^n$ as shown in section~\ref{sec:ti};
    \item get $\V{Q}_0$ and $\V{Q}_3$ from \eqref{eq:q03};
    \item calculate $\V{P}_0$ and $\V{P}_2$;
    \item compute $\{\V{C}_i\}_{i=0}^n$ with \eqref{eq:Ci};
    \item calculate $D_0$, $D_2$, $E_{01}$, $E_{02}$, $E_{11}$,
	$E_{12}$, $E_{13}$, $E_{22}$ and $E_{23}$ with \eqref{eq:DE};
    \item calculate $A_1$, $A_2$, $A_3$, $A_4$ and $A_5$ with
	\eqref{eq:ACD};
    \item calculate $r$ and $s$ with \eqref{eq:rs};
    \item get $\V{Q}_1$ and $\V{Q}_2$ from \eqref{eq:q12}.
\end{enumerate}

$\V{Q}_0$ and $\V{Q}_3$ are respectively the starting and ending
points of the offset Bézier curve while $\V{Q}_1$ and $\V{Q}_2$ are
its control points.

\section{Choosing $t_i$\label{sec:ti}}

To select the $\{t_i\}_{i=0}^n$ set of values for $t$ needed by
the offsetting algorithm, we can use different methods. Here are some
basic ones: no further research is performed to check the quality of
the results.

\subsection{Method 1: too lazy to think}

The most obvious method is to directly use evenly spaced time values:

\begin{equation*}
    t_i = \frac{i}{n}.
\end{equation*}

\subsection{Method 2: squared distances}

Let's select some $\{\V{D}_i\}$ points on $\V{B}(t)$, for instance by
resolving the $t$ values got from the lazy method. The following formula
will partition the Bézier curve proportionally to their squared
distances:

\begin{alignat*}{2}
    t_0 &= 0; &\qquad d = \sum_{i=1}^n \left( \V{D}_i - \V{D}_{i-1} \right)^2 \\[-20pt]
    t_i &= t_{i-1} + \frac{\left( \V{D}_i - \V{D}_{i-1} \right)^2}{d}. \\
\end{alignat*}

\subsection{Method 3: distances}

A variant of the previous method that uses distances instead of squared
distances. This is computationally more intensive because the norm of
a vector $ \| \V{D} \| \equiv \sqrt{D_x^2 + D_y^2}$ requires a square
root.

\begin{alignat*}{2}
    t_0 &= 0; &\qquad d = \sum_{i=1}^n \| \V{D}_i - \V{D}_{i-1} \| \\[-16pt]
    t_i &= t_{i-1} + \frac{\| \V{D}_i - \V{D}_{i-1} \|}{d}. \\
\end{alignat*}

\end{document}
